Bioteknikdagarna 2014

Varje år anordnas Bioteknikdagarna, BTD, av organisationen Bioteknikstudenterna, BTS. Syftet är att förena, marknadsföra och främja samarbetet mellan Sveriges bioteknikutbildningar. Studenter från Linköpings Tekniska Högskola, Lunds Tekniska Högskola, Umeå universitet, Uppsalas universitet, Kungliga Tekniska Högskolan och Chalmers samlas årligen på ett av lärosätena för att delta i Bioteknikdagarna. 

När senaste upplagan gick av stapeln den 2-5 oktober 2014 var det Linköpings tur att vara värd för festligheterna. Då fick TBi-sektionen nöjet att anordna en långhelg fylld av trevligheter.  En projektgrupp bestående av tio engagerade studenter tog sig an uppdraget att genomföra just detta.

BTD 2014 blev en lyckad tillställning där ungefär 90 studenter deltog. Bioteknikdagarna inleddes på torsdagskvällen med en fulsittning där studenterna från de olika lärosätena kunde visa upp sina varierade sittnings- och gückeltraditioner. TBi-sektionen bjöd på gückel från bland annat Urfadderiet och sektionens två egna gückelgrupper QQQ och Phitau.

Under de kommande dagarna fick studenterna möjligheten att på en företagsmässa träffa företag inom branschen och gå på flera inspirerande föreläsningar med anknytning till utbildningen. Utöver detta var Bioteknikdagarna ett ypperligt tillfälle att träffa studenter på liknande utbildningar för att utbyta erfarenheter och knyta kontakter. 

Avslutningsvis hölls en finsittning där studenterna fick tillfälle att klä upp sig lite extra och mingla med sina nyvunna vänner innan färden tillbaka till respektive lärosäte. 

\image{img/bioteknikdagarna.jpg}{Projektgrupp BTD 2014.}

Bildkälla: http://www.bioteknikdagarna.org/contact.html