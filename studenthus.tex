Nytt studenthus 2018

Drömmen om ett nytt studenthus, med ännu fler studieplatser och där all service för studenterna är samlad, kommer nu att gå i uppfyllelse. På platsen där nuvarande Origo står planeras ett sexvåningshus stå klart för inflyttning september 2018.

Målet är att studenthuset ska bidra till ett större samarbete mellan universitetet och samhället men även samla servicefunktioner för studenterna. Istället för att servicefunktionerna är utspridda på sju olika hus, som det är i nuläget, kommer exempelvis reception, studenthälsan och antagnings- och examensservice samlas på en och samma plats. Det kommer även finnas fler studieplatser, ett studentfik, salar till konferenser eller föredrag och ett sprillans nytt bibliotek. Ett hus för studenterna helt enkelt!
För att öka samarbetet mellan universitetet och samhället kommer en viss verksamhet från Mjärdevi Science Park flytta till Campus Valla. Huset planeras bli byggt med innovativa lösningar för att uppfylla höga krav på energieffektivitet. Det gamla huset Origo kommer att monteras ned på ett hållbart sätt där stora delar av huset kan återbrukas. Satsningen kommer att gå på ca 400 miljoner kronor och kommer verkligen visa att universitetet ligger i framkant och är ett attraktivt val för nya studenter.
 
\image{img/studenthus_inne.jpg}{Inne.}
\image{img/studenthus_ute.jpg}{Ute.}