

\begin{editorial}{3}{Ledare}{Omid Philip Thern}{label}



Det finns mycket man har
roligt åt under året och mycket som hunnits med. Att vara aktiv inom
alumniutskottet ger mig massa bra erfarenheter och roliga kontakter
både inom och utanför sektionen. Utskottet jobbar hela tiden på att
motivera studenter till att göra det de gör bäst, studera. Som
tidigare år startades verksamhetsåret med en härlig kickoff för
mentorskapprogrammet. 

Dessförinnan det anordnades det en
lunchföreläsning med årets TBI-alumn Tove Ahlström som du kan läsa mer
om längre fram i tidningen. Det sista som kommer att avsluta hela året
är en pub i Stockholm som hålls för alla alumner som vill återträffas
och prata gamla minnen.

\image{img/ledarebild.jpg}

I vårens upplaga av tidningen tas inte bara intressanta saker som iGEM
och våra kära bioteknikdagar som i år hölls på Linköpings Universitet
utan även om hur sektionens alla olika delar som livar upp och styr
sektionen ser ut i dagslägget. Styret ger oss en inblick i hur och vad
sektionen håller på med nuförtiden och Bi6-chefen visar hur festerna
går till. Vi fortsätter självklart som vanligt att skriva om
artikelserien om forskning på LiU. Avslutningsvis ges en artikel om
ombyggnationen av det kända huset Origo som förhoppningsvis kommer att
starta under hösten 2015.





\end{editorial}