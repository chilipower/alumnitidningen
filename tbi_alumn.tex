\begin{news}{2}{Årets TBi-Alumn}{}{}{6}{}

\image{img/tbi_alumni_t.png}{\\Tove Ahlström}

\emph{Årets TBi-alumn} en av de nyare traditionerna som sektionen har där vi varje 
år ger både alumner och studenter möjlighet att nominera alumner som de 
tycker har gjort något utöver det vanliga. \emph{Tove Ahlström} blev årets alumn 
2014 och valdes på grund av hennes stora engagemang inom miljö- och 
klimatfrågor som hon inte bara jobbar med utan också håller föredrag om 
utanför arbetet. Tove kommer från Huddinge och började studera på Linköpings 
universitet för 18 år sedan vilket också var det första året då Urfadderiet 
tog över nollningen av studenterna.
\\
 
\noindent \textbf{Var gjorde du ditt ex-jobb?}
 
 - Under tiden när jag gick på universitetet så studerade man tre och ett halvt 
 år som vanlig utbildning och sedan fick man välja ett profilblock, som det 
kallades för då. Profilblock är det som idag kallas mastersutbildning och 
precis som de olika mastersprofilerna fanns det olika profilblock man kunde 
välja mellan. Det jag valde var \emph{storskalig biologisk produktion}. Man fick 
söka själv efter sitt ex-jobb genom att höra av sig till de olika företagen 
som fanns inom sin inriktning. Jag fick napp ganska så snabbt på ett företag 
som hette \emph{Biovitrum}, numera \emph{Sobivitrum}, där jag arbetade med en kompis från 
universitetet med att utveckla en metod som framställde olika protein ur 
celler från människor.

Processen som vi arbetade på gick ut på att odla humana embryoligiska celler 
i suspension med hjälp av tillväxthormoner som tillsattes. Tillsammans med 
att cellerna växte så transfekterades nya DNA-strängar in i cellerna för att 
de skulle kunna tillverka det protein man ville få fram. Idén med att 
vi använde humana celler var för att undvika problemet med att bakterieceller 
inte producerar proteiner i glykosylerad form vilket humana celler gör. 
Glykosylerade protein ses inte som främmande för människokroppen. När hela 
processen var framtagen var tanken att man skulle kunna skapa vilket protein 
som helst beroende på vilken DNA-sekvens man gav de humana cellerna.
\\
 
\noindent \textbf{Har ex-jobbet varit relevant för dig när du tog examen?}

 - Ja, det var en länk in i arbetslivet. Idag jobbar jag inte alls med bioteknik 
utan för det mesta med miljöfrågor. Men när jag tog examen så var ex-jobbet 
viktigt för mig. Ex-jobbet gav mig möjlighet att få fortsatt anställning på 
Biovitrum. De ville att jag skulle fortsätta arbeta på det som utgjorde mitt 
ex-jobb som en typ av projektanställning. Efter det och allt annat som hänt 
mig så hamnade jag där jag är idag. 
\\

\noindent \textbf{Vad var det roligast med att plugga?}

 - Hela grejen är att alla kommer från olika ställen och ingen har mycket mer än 
skolan och sin lägenhet. Det skapade på något sätt en gemensam grund för alla 
studenter som vi därefter byggde en gemenskap på. Det är självklart så att 
jag har formats väldigt mycket av min studietid och många av mina 
studiekompisar har jag kvar som nära vänner idag. Vi hade alla liknade liv; 
att ta en kaffe i \emph{Baljan} att cykla till skolan i motvind under tentaperioden 
eller anordna festligheter med \emph{Kaspers Klubb} på NH. Det finns inte mycket mer 
man kan säga än att det var en utav de bästa perioderna av mitt liv så långt. 
Det var också en helt ny tid för sektionen då jag började. Utbildningen 
teknisk biologi hade brutit sig loss ur Maskinsektionen och skapade den 
sektion vi har idag med Bi6, Urfadderiet och alla andra delar som gör 
sektionen så härlig som den är.  
\\

\noindent \textbf{Vad gör du i dagsläget?}

Jag jobbar på \emph{Karolinska Universitetssjukhuset} som miljöhandläggare med 
läkemedels- och klimatfrågor och styr flera projekt som har kopplingar till 
mina områden. Jag är också \emph{Climate Leader} och företräder \emph{Climate 
Reality Project}, ett klimatprojekt som Al Gore startade upp efter att filmen \emph{''En 
obekväm sanning''} hade premiär 2006. Att vara Climate Leader betyder att 
man har fått en utbildning i hur och vad man ska föreläsa om av Al Gore 
själv. Efter att man har haft utbildningen får man tillstånd att hålla 
föredrag med Al Gores material för andra. Man är inte anställd på något sätt 
utan arbetar som volontär. Förutom jag så finns det ungefär sjutusen andra 
föreläsare i världen och i Sverige är jag nog den enda som håller dessa 
föredrag om klimatförändringarna. 

\end{news}
