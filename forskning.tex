\begin{news}{3}{Forskning på LiU}{Noggrann detektion av proteinstrukturer hoppas kunna bidra till kampen mot malaria.}{}{}


Alumniutskottet har pratat med Patrik Lundström som är universitetslektor vid institutionen för fysik, kemi och biologi (IFM) vid Linköpings universitet. Där leder han en forskningsgrupp som består av en postdoc, två doktorander och exjobbare. För tillfället deltar en masterstudent och två kandidatstudenter i forskningsgruppen. 

Det som ligger till grund för två av gruppens huvudsakliga projekt är studier kring \emph{kinaser}, med fokus på deras dynamik. Det är bland annat dynamiken vid uppveckning, ligandbindning och hastigheter som studeras, vilket är betydande för proteiners funktion. 

För att studera dynamiken hos kinaser tillämpas i huvudsak \emph{NMR}, kärnmagnetisk resonansspektroskopi. Med denna metod fås fingertryck av proteinet, till exempel om det är uppveckat eller ouppveckat. Vid uppföljning av experimentet kan samma slags experiment utföras nästa vecka för att se om proteinet har förändrats. 

Ett annat sätt som NMR används är vid tillordning av proteinet, då varje enskild NMR-signal kopplas till en viss atom. Genom att studera var topparna ligger i spektrumet fås även kunskap om proteinets struktur.  

\image{img/forskning_protein.png}{Strukturen för proteinet EphB2 som Patriks forskningsgrupp undersöker i ett av sina projekt vid LiU.}

I ett av projekten arbetar gruppen med kinaset \emph{EphB2}. Proteinet är väsentligt för att nervceller ska hitta rätt under fosterutveckling samt för tillväxten av blodkärl. Om proteinet inte fungerar som det ska kan det leda till cancer och alzheimers sjukdom vilket gör det till ett högaktuellt forskningsområde.

Det andra projektet fokuserar på \emph{malaria} som är en allvarlig och i många fall dödlig sjukdom som tar livet av ungefär en halv miljon människor om året; mestadels barn. Sjukdomen orsakas av parasiter av släktet \emph{Plasmodium} som sprids via blodsugande myggor. När en mygga suger blod hamnar parasiten i människans blodomlopp. Dessa tar sig vidare till levern där dotterceller börjar produceras vilka därefter släpps ut i blodet där de tar sig in i blodkroppar. Väl inne bryter de ner hemoglobinet för att kunna tillgodogöra sig näringen vilket i slutändan förstör blodkropparna. Som följd av detta börjar den infekterade lida av anemi och andra åkommor som feber, frossa och illamående. 

\image{img/forskning_oversikt.jpg}{Livscykel för malariaparasiten}


Projektet lägger fokus på att karaktärisera ett kalciumberoende proteinkinas kallat \emph{CDPK3} som förekommer hos malariaparasiten. Det är dessa kinaser som ansvarar för en del nyckelmoment i malaria-processen. 
För båda projekten är det långsiktiga målet främst en biologisk frågeställning där fokus ligger på att exempelvis hitta en molekyl som kan utgöra en del av läkemedelsutvecklingen. För malariaprojektet hoppas därmed projektgruppen att kunna utveckla en inhibitor till CDPK3, vilket därmed skulle kunna leda till utveckling av ett funktionellt läkemedel.

Projektet med EphB2 har pågått i några år och resultaten har visat roliga saker om hur kinaset skulle kunna fungera. Till exempel bildar den aktiva formen av kinasddomänen \emph{dimerer}, vilket kan ha stor betydelse för hur vidare signalering sker. Projektgruppen har även hittills funnit en hel del spännande egenskaper hos malariaproteinet, bland annat att det växlar mellan två olika olika strukturer ungefär tusen gånger varje sekund. De har goda förhoppningar om att kunna publicera resultat inom en snar framtid.

Patrik påbörjade sin resa i Lund där han studerade till civilingenjör i kemiteknik. Efter detta fortsatte han på Biomedicinska forskarskolan trots att han egentligen hade en mer fysikalisk inriktning. Där deltog han i tre projekt varav ett av dem utfördes på avdelningen för molekylärmedicin i genterapi. I projektet undersöktes blodbindning hos transgena möss, vilket han tyckte var intressant och han fick senare en doktorandtjänst vid samma avdelning. Efter två år som doktorand på avdelningen för molekylärmedicin i genterapi kände Patrik att han inte hade hamnat rätt. Han valde därför att istället söka sig till fysikalisk kemi där han doktorerade i fem år. Under dessa år arbetade han med metodutveckling av NMR. När han 2005 var klar med sin doktorandtjänst fick han en postdoc tjänst i Toronto där han huvudsakligen arbetade med utveckling av metoder för att strukturbestämma osynliga tillstånd av proteiner, det vill säga de strukturer som proteiner oftast inte befinner sig i. År 2009 fick Patrik en FOASS tjänst vid IFM och startade sin forskargrupp. Sedan 2 år tillbaka har Patrik Lundström idag fast anställning och pågående forskning.

\end{news}