iGEM, International Genetically Engineered Machine är en stor
internationell tävling inom syntetisk biologi. Studenterna får själva
utforma ett projekt baserat på användandet av biobricks, en typ av
standardiserade DNA-sekvenser kompatibla med restriktionsenzym. Dessa
kan infogas i E.coli eller andra levande celler för att skapa
”biologiska maskiner”. Tävlingen som startades av MIT har arrangerats
i över tio år och fortsätter att växa. År 2014 deltog 245 lag från 50
olika länder. Från Sverige deltar utöver Linköpings universitet även
Uppsala och Chalmers.


\image{img/igem_gruppfoto.png}{Deltagare från LiU iGEM 2014 samlade
framför deras monter under den stora internationella ”iGEM Jamboree” i
Boston, USA.}

\image{img/igem_patricia.png}{Patricia Gullberg.}


Linköpings universitet deltog för första gången år 2013. Projektet
handlade om detektion av äggallergener i mat med hjälp av en detektor
baserad på antikroppar och FRET, fluorescens resonance energy
transfer. Laget från 2014 fortsatte sedan på liknande spår då man
fokuserade på detektion av jordnötsallergener. Laget som bestod av 15
medlemmar, de flesta från TBi åkte under hösten till MIT, Boston och
presenterade sitt projekt på Giant Jamboree.  -Syftet med iGEM är att
nå ut till allmänheten för att ge dem en bättre förståelse om vad
syntetisk biologi faktiskt handlar om och vad man kan göra, säger
Patricia Gullberg som deltog i LiU iGEM-laget 2014 och är dessutom
handledare för årets lag.  - Det handlar även om att dela kunskap
mellan varandra inom forskningsvärlden, berättar Patricia. Just nu
läser hon masterprogrammet Protein Science vid LiU efter att hon
examinerades från KBnv. Vi fick tag på henne för en intervju där hon
berättar om sina erfarenheter i iGEM.

Vad är läget för årets lag?

- Laget har precis blivit sammansatt och just nu hålls många möten för
  att få in de nya deltagarna i projektet då vi har valt att fortsätta
  på samma spår som förra året, men förmodligen med en del
  förändringar. Vi kommer även jobba mycket med att brainstorma,
  planera och förbereda mycket inför sommarens arbete den närmaste
  tiden.

Vad fick dig att vilja vara med? Vad fick dig att även vara med ett andra år?

- Jag tyckte att tävlingen lät väldigt rolig och givande och jag såg
  en chans att utveckla mina kunskaper ännu mer. Jag tyckte också det
  verkade väldigt häftigt med att åka till Boston för att tävla och
  träffa massa andra människor som brinner för samma sak. Att jag
  valde att vara med ytterligare ett år till beror på att det var så
  sjukt kul och givande och jag brinner för projektet så jag kunde
  inte riktigt släppa det. När jag blev erbjuden en anställning som
  handledare kunde jag inte tacka nej.

Härligt! Vad har du fått ut av tävlingen?

- Jag har fått många nya goda vänner, jag har utvecklats som person
  och känner mig självsäker på lab på ett helt annat sätt nu. Då detta
  är ett studentdrivet projekt är det vi studenter som gör allting
  från grunden. Då man tvingas utmana sig själv och läsa på för att
  förstå hur man ska lösa ett problem så utvecklar man verkligen sitt
  problemtänkande och man vågar ta för sig mer.

Sammanfattningsvis, vad är det bästa med iGEM?

- Att man får lära känna nya människor och utnyttja sina kunskaper på
  ett väldigt stimulerande sätt!



För mer information:

iGEMs oficiella hemsida
https://www.igem.org/Main_Page

Förra årets bidrag LiUiGEM 2014
http://2014.igem.org/Team:Linkoping_Sweden

Artikel i lokaltidningen Corren
http://www.corren.se/nyheter/linkoping/biologistudenter-vill-hjalpa-jordnotsallergiker-7151455.aspx
