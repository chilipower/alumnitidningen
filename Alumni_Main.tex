\documentclass[10pt,final,hyphenatedtitles]{papertex}




\usepackage[utf8]{inputenc}
\usepackage[swedish]{babel}
\usepackage[T1]{fontenc}
\usepackage{ulem}
%\usepackage{times}
\usepackage{todonotes}

%\usepackage{color}
%\usepackage[usenames,dvipsnames]{xcolor}





%\definecolor{color}{cmyk}{1, 0,0, 0.5}

%\renewcommand{\indexEntryFormat}{\large\rmfamily}
%\renewcommand{\indexEntryPageTxt}{page}
%\renewcommand{\timestampSeparator}{$\Rightarrow$}
%\renewcommand{\innerTextFinalMark}{$\spadesuit$}


\renewcommand{\logo}{\mylogo{\noindent\hrulefill{\fontsize{10mm}{14mm} \usefont{T1}{pag}{m}{n} \textcolor{black}{ALUMNITIDNINGEN}}\\[3pt]}}
%\renewcommand{\editionFormat}{\LARGE}
%\renewcommand{\indexEntryFormat}{\normalsize\rmfamily}

\author{Alumniutskottet}
\title{my first paperTeX}
\edition{Till alla TBi-sektionens alumner - Störst, bäst och vackrast!}
%\tolerance=300




\begin{document}

\begin{frontpage}
%\todo{Ledare stavningskoll}

\begin{editorial}{3}{Ledare}{Omid Philip Thern}{label}
Det finns mycket man har
roligt åt under året och mycket som hunnits med. Att vara aktiv inom
alumniutskottet ger mig massa bra erfarenheter och roliga kontakter
både inom och utanför sektionen. Utskottet jobbar hela tiden på att
motivera studenter till att göra det de gör bäst, studera. Som
tidigare år startades verksamhetsåret med en härlig kickoff för
mentorskapprogrammet.

Dessförinnan det anordnades det en
lunchföreläsning med årets TBI-alumn Tove Ahlström som du kan läsa mer
om längre fram i tidningen. Det sista som kommer att avsluta hela året
är en pub i Stockholm som hålls för alla alumner som vill återträffas
och prata gamla minnen.

\image{img/ledarebild.jpg}

I vårens upplaga av tidningen tas inte bara intressanta saker som iGEM
och våra kära bioteknikdagar som i år hölls på Linköpings Universitet
utan även om hur sektionens alla olika delar som livar upp och styr
sektionen ser ut i dagslägget. Styret ger oss en inblick i hur och vad
sektionen håller på med nuförtiden och Bi6-chefen visar hur festerna
går till. Vi fortsätter självklart som vanligt att skriva om
artikelserien om forskning på LiU. Avslutningsvis ges en artikel om
ombyggnationen av det kända huset Origo som förhoppningsvis kommer att
starta under hösten 2015.
\end{editorial}

\todo{Ledare stavningskoll}

\firstimage{img/forskning_protein.png}{På bilden visas EphB2, ett protein som utgör ett av forskningsgruppen projekt lett av universitetslektorn Patrik Lundström.}
\firstnews{Forskning vid LiU: Proteinets struktur avslöjar hur sjukdomar kan motverkas.}{Universitetslektor Patrik Lundström och hans forskningsteam undersöker på detaljnivå hur kinaser och dess struktur är kopplat till sjukdomar som malaria, cancer och alzheimers. Med NMR (nuclear magnetic resonance) ökar förhoppningarna till att hitta en måltavla för läkemedelsutvecklingen. Läs mera om denna intressanta forskning på sida 3.}
{1}

\secondnews{Origo skall rivas för att ge plats till ett sprillans nytt studenthus.}{Ett luftigt sexvåningshus skall utgöra samlingspunkt för service, studenter samt ett ökat samarbete med samhället. Bygget planeras vara klart för inflyttning år 2018.}{Nyligen beslutades att Linköpings universitet skall satsa på en ny byggnad mitt i Campus Valla. Ett studentfik, studieplatser, ett nytt bibliotek samt en viss verksamhet från Mjärdevi Science Park är bara några exempel av vad vi kan förvänta oss när bygget står klart. Alumniutskottet är förväntansfulla på slutresultatet! Läs mer och kolla in bilderna på sida 5.}{Sida 5/6}{2}

\thirdnews{Tredje året i rad för Linköpings iGEM-lag.}{Ännu ett lag har satts ihop inför årets världsomspännande tävling i systembiologi. Handledare Patricia Gullberg berättar om hennes engagemang i LiU iGEM.}{- Man tvingas utmana sig själv och läsa på för att
  förstå hur man ska lösa ett problem, berättar handledare och förra årets deltagare Patricia. iGEM går ut på att studenter inom bland annat biologi, teknik och kemi går ihop för att med biobricks skapa innovativa projekt inom hälsa, miljö, teknik och mycket mer. En stor del av tävlingen är att sprida kunskap om systembiologi till samhället. Även en stor gnutta kreativitet hör till iGEM-andan. Läs mer av intervjun på sida 4.}{img/igem_logo}{Sida 4/6}{3}{}

\begin{indexblock}{I detta nummer...}

\indexitem{Ledare med vår käre alumniansvarige Omid Philip Thern}{1}

\indexitem{Tre snabba: TBi-ordförande Mona Mrad har ordet.}{2}

\indexitem{Bi6 om året som gått med Caroline Blomström}{2}

\indexitem{Bioteknikdagarna på LiU: En lyckad tillställning med mingel, mässa och sittingar.}{4}


\end{indexblock}

\begin{authorblock}

\textbf{ALUMNIUTSKOTTET}

Omid Philip Thern\\
Elisabet Bengts\\
Julia Bjers\\
Victoria Isabelle Boman\\
Linda Gidlöf\\
Lovisa Hemmingsson\\
Elina Jansson\\
Annie von Scheele\\

\end{authorblock}


\begin{weatherblock}{VÄDRET CAMPUS VALLA}
%\weatheritem{img/weather/bob_sol.png}{IDAG}{2}{2}{}
%\weatheritem{img/weather/bob_sol.png}{TODAY}{13}{9}{}
\end{weatherblock}





\end{frontpage}
\clearpage

\newsection{}






\todo{Ledare stavningskoll}

\begin{editorial}{3}{Ledare}{Omid Philip Thern}{label}
Det finns mycket man har
roligt åt under året och mycket som hunnits med. Att vara aktiv inom
alumniutskottet ger mig massa bra erfarenheter och roliga kontakter
både inom och utanför sektionen. Utskottet jobbar hela tiden på att
motivera studenter till att göra det de gör bäst, studera. Som
tidigare år startades verksamhetsåret med en härlig kickoff för
mentorskapprogrammet.

Dessförinnan det anordnades det en
lunchföreläsning med årets TBI-alumn Tove Ahlström som du kan läsa mer
om längre fram i tidningen. Det sista som kommer att avsluta hela året
är en pub i Stockholm som hålls för alla alumner som vill återträffas
och prata gamla minnen.

\image{img/ledarebild.jpg}

I vårens upplaga av tidningen tas inte bara intressanta saker som iGEM
och våra kära bioteknikdagar som i år hölls på Linköpings Universitet
utan även om hur sektionens alla olika delar som livar upp och styr
sektionen ser ut i dagslägget. Styret ger oss en inblick i hur och vad
sektionen håller på med nuförtiden och Bi6-chefen visar hur festerna
går till. Vi fortsätter självklart som vanligt att skriva om
artikelserien om forskning på LiU. Avslutningsvis ges en artikel om
ombyggnationen av det kända huset Origo som förhoppningsvis kommer att
starta under hösten 2015.
\end{editorial}


\newssep

\begin{shortnews}{3}{Tre snabba med Ordförande 14/15}{Mona berättar!}{}
\shortnewsitem{Hur har du upplevt året?}{Året har varit helt super! Detta
verksamhetsår har varit lite annorlunda från tidigare år då styrelsen
nu endast består av sju ledamöter. Tillsammans så har vi kämpat hårt
med diverse frågor som fört sektionen framåt och det ska bli riktigt
tråkigt att få lämna det vi byggt upp.}

\shortnewsitem{Har det skett några förändringar sedan tidigare år?}{Den största förändringen är utökade platser i kansliet. De flesta utskottsledare
sitter nu i kansliet och inte i styrelsen som förr. Vi har även bytt
namn på våra sektionsmöten så att namnet överensstämmer med årstiden!
Vårmöte 1 heter numera Wintermöte, W på grund av att sittande vice
heter Winter i efternamn, och Vårmöte 2 heter endast Vårmöte. ÄNTLIGEN är sektionsväskor beställda och i höst kommer flera TBi-studenter bära
väskan med stolthet!}

\shortnewsitem{Har du någon rolig anekdot från året?}{Under höstmötet så la styrelsen fram flera propositioner angående viktiga stadgeförändringar. Det tog
ca 10 minuter att besluta om dessa proppar som skulle förändra en del
av sektionens struktur och sektionens framtid. Resten av tiden gick
åt för att diskutera vilken färg dragkedjan på de nya sektionsväskorna
skulle ha.}



Mona Mrad
Ordförande 14/15

\end{shortnews}


\newssep

\begin{news}{3}{Bi6 visar hur man gör!}{}{}{3}

Sektionens festeri, Bi6, hålls fortfarande vid liv och likt tidigare år var
vi tolv glada själar som startade året med grym energi! Vi har anordnat olika
sittningar och aktiviteter för sektionen sedan maj förra året. Vår ambition med
året har varit att involvera fler personer från sektionen i våra event. Vi vill
anordna sittningar och fester som passar alla! 

Vi har även inlett ett lite
tajtare samarbete med läkarsektionens festeri, MedSex, vilket varit både roligt
och givande. Vi har hjälpt varandra nå ut till elever på de olika campusen och
även lärt mycket av varandra! Såklart så gick även Snuttefilmen av stapeln i
februari där vi likt tidigare år hade vår chans att brilljera, och det gjorde
vi! Årets tema var Game of Thrones - ingen film kanske man tänker. Vi valde att
gå över till att ha en serie som tema då det kändes väldigt aktuellt. Vi fick
grym respons av universitetet och många frågade när den riktiga filmen skulle
släppas, som att vår introfilm skulle ha varit en trailer. Riktigt kul!

Vi sålde slut till båda Kårallenfesterna men tyvärr hade vi några biljetter 
kvar till den tidiga filmvisningen. I övrigt gick Snuttefilmen kalasbra och 
blev en hel vecka tack vare GörU som anordnade en filmpub på onsdagen. Det har 
varit otroligt roligt att jobba i en så bra grupp och att komma många härliga
personer så nära. Nu laddar vi inför nya utmaningar och lämnar tryggt över till
Bi6 15/16 som är nyuttagna. Vi är redo att bli pateter!
\\

\image{img/bi6.jpg}{\\Caroline Blomström, Bi6-chef 14/15.\\ \\ \\}


\end{news}


\newssep

\begin{news}{3}{Forskning på LiU}{Noggrann detektion av proteinstrukturer hoppas kunna bidra till kampen mot malaria.}{}{}


Alumniutskottet har pratat med Patrik Lundström som är universitetslektor vid 
institutionen för fysik, kemi och biologi (IFM) vid Linköpings universitet. 
Där leder han en forskningsgrupp som består av en postdoc, två doktorander 
och exjobbare. För tillfället deltar en masterstudent och två 
kandidatstudenter i forskningsgruppen. 

Det som ligger till grund för två av gruppens huvudsakliga projekt är studier 
kring \emph{kinaser}, med fokus på deras dynamik. Det är bland annat 
dynamiken vid uppveckning, ligandbindning och hastigheter som studeras, 
vilket är betydande för proteiners funktion. 

För att studera dynamiken hos kinaser tillämpas i huvudsak \emph{NMR}, 
kärnmagnetisk resonansspektroskopi. Med denna metod fås fingertryck av 
proteinet, till exempel om det är uppveckat eller ouppveckat. Vid uppföljning 
av experimentet kan samma slags experiment utföras nästa vecka för att se om 
proteinet har förändrats. 

Ett annat sätt som NMR används är vid tillordning av proteinet, då varje 
enskild NMR-signal kopplas till en viss atom. Genom att studera var topparna 
ligger i spektrumet fås även kunskap om proteinets struktur.  

\image{img/forskning_protein.png}{Strukturen för proteinet EphB2 som Patriks forskningsgrupp undersöker i ett av sina projekt vid LiU.}

I ett av projekten arbetar gruppen med kinaset \emph{EphB2}. Proteinet är 
väsentligt för att nervceller ska hitta rätt under fosterutveckling samt för 
tillväxten av blodkärl. Om proteinet inte fungerar som det ska kan det leda 
till cancer och alzheimers sjukdom vilket gör det till ett högaktuellt 
forskningsområde.

Det andra projektet fokuserar på \emph{malaria} som är en allvarlig och i 
många fall dödlig sjukdom som tar livet av ungefär en halv miljon människor 
om året; mestadels barn. Sjukdomen orsakas av parasiter av släktet \emph
{Plasmodium} som sprids via blodsugande myggor. 

När en mygga suger blod hamnar parasiten i människans blodomlopp. Dessa tar 
sig vidare till levern där dotterceller börjar produceras vilka därefter 
släpps ut i blodet där de tar sig in i blodkroppar. Väl inne bryter de ner 
hemoglobinet för att kunna tillgodogöra sig näringen vilket i slutändan 
förstör blodkropparna. Som följd av detta börjar den infekterade lida av 
anemi och andra åkommor som feber, frossa och illamående. 

\image{img/forskning_oversikt.jpg}{Livscykel för malariaparasiten}


Projektet lägger fokus på att karaktärisera ett kalciumberoende proteinkinas 
kallat \emph{CDPK3} som förekommer hos malariaparasiten. Det är dessa kinaser 
som ansvarar för en del nyckelmoment i malaria-processen. 
För båda projekten är det långsiktiga målet främst en biologisk 
frågeställning där fokus ligger på att exempelvis hitta en molekyl som kan 
utgöra en del av läkemedelsutvecklingen. För malariaprojektet hoppas därmed 
projektgruppen att kunna utveckla en inhibitor till CDPK3, vilket därmed 
skulle kunna leda till utveckling av ett funktionellt läkemedel.

Projektet med EphB2 har pågått i några år och resultaten har visat roliga 
saker om hur kinaset skulle kunna fungera. Till exempel bildar den aktiva 
formen av kinasddomänen \emph{dimerer}, vilket kan ha stor betydelse för hur 
vidare signalering sker. Projektgruppen har även hittills funnit en hel del 
spännande egenskaper hos malariaproteinet, bland annat att det växlar mellan 
två olika olika strukturer ungefär tusen gånger varje sekund. De har goda 
förhoppningar om att kunna publicera resultat inom en snar framtid.

Patrik påbörjade sin resa i Lund där han studerade till civilingenjör i 
kemiteknik. Efter detta fortsatte han på Biomedicinska forskarskolan trots 
att han egentligen hade en mer fysikalisk inriktning. Där deltog han i tre 
projekt varav ett av dem utfördes på avdelningen för molekylärmedicin i 
genterapi. I projektet undersöktes blodbindning hos transgena möss, vilket 
han tyckte var intressant och han fick senare en doktorandtjänst vid samma 
avdelning.
 
Efter två år som doktorand på avdelningen för molekylärmedicin i 
genterapi kände Patrik att han inte hade hamnat rätt. Han valde därför att 
istället söka sig till fysikalisk kemi där han doktorerade i fem år. Under 
dessa år arbetade han med metodutveckling av NMR. När han 2005 var klar med 
sin doktorandtjänst fick han en postdoc tjänst i Toronto där han 
huvudsakligen arbetade med utveckling av metoder för att strukturbestämma 
osynliga tillstånd av proteiner, det vill säga de strukturer som proteiner 
oftast inte befinner sig i. 

År 2009 fick Patrik en FOASS tjänst vid IFM och 
startade sin forskargrupp. Sedan 2 år tillbaka har Patrik Lundström idag fast 
anställning och pågående forskning.

\end{news}

\newssep

Bioteknikdagarna 2014

Varje år anordnas Bioteknikdagarna, BTD, av organisationen
Bioteknikstudenterna, BTS. Syftet är att förena, marknadsföra och
främja samarbetet mellan Sveriges bioteknikutbildningar. Studenter
från Linköpings Tekniska Högskola, Lunds Tekniska Högskola, Umeå
universitet, Uppsalas universitet, Kungliga Tekniska Högskolan och
Chalmers samlas årligen på ett av lärosätena för att delta i
Bioteknikdagarna.

När senaste upplagan gick av stapeln den 2-5 oktober 2014 var det
Linköpings tur att vara värd för festligheterna. Då fick TBi-sektionen
nöjet att anordna en långhelg fylld av trevligheter.  En projektgrupp
bestående av tio engagerade studenter tog sig an uppdraget att
genomföra just detta.

BTD 2014 blev en lyckad tillställning där ungefär 90 studenter
deltog. Bioteknikdagarna inleddes på torsdagskvällen med en
fulsittning där studenterna från de olika lärosätena kunde visa upp
sina varierade sittnings- och gückeltraditioner. TBi-sektionen bjöd på
gückel från bland annat Urfadderiet och sektionens två egna
gückelgrupper QQQ och Phitau.

Under de kommande dagarna fick studenterna möjligheten att på en
företagsmässa träffa företag inom branschen och gå på flera
inspirerande föreläsningar med anknytning till utbildningen. Utöver
detta var Bioteknikdagarna ett ypperligt tillfälle att träffa
studenter på liknande utbildningar för att utbyta erfarenheter och
knyta kontakter.

Avslutningsvis hölls en finsittning där studenterna fick tillfälle att
klä upp sig lite extra och mingla med sina nyvunna vänner innan färden
tillbaka till respektive lärosäte.

\image{img/bioteknikdagarna.jpg}{Projektgrupp BTD 2014.}

Bildkälla: http://www.bioteknikdagarna.org/contact.html


\newssep




iGEM, International Genetically Engineered Machine är en stor internationell tävling inom syntetisk biologi. Studenterna får själva utforma ett projekt baserat på liknande verktyg i form av biobricks, en typ av standardiserade DNA-sekvenser kompatibla med restriktionsenzym. Dessa kan sedan infogas i E.coli eller andra levande celler för att skapa ”biologiska maskiner”. Tävlingen som startades av MIT har arrangerats i över tio år och fortsätter att växa. År 2014 deltog 245 lag från 50 olika länder. Från Sverige deltar utöver Linköpings universitet även Uppsala och Chalmers.



\image{img/igem_gruppfoto.png}{Deltagare från LiU iGEM 2014 samlade framför deras monter under den stora internationella ”iGEM Jamboree” i Boston, USA.}

\image{img/igem_patricia.png}{Patricia Gullberg.}


Linköpings universitet deltog för första gången år 2013. Projektet handlade om detektion av äggallergener i mat med hjälp av en detektor baserad på antikroppar och FRET, fluorescens resonance energy transfer. Laget från 2014 fortsatte sedan på liknande spår då man fokuserade på detektion av jordnötsallergener. Laget som bestod av 15 medlemmar, de flesta från TBi åkte under hösten till MIT, Boston och presenterade sitt projekt på Giant Jamboree. 
-Syftet med iGEM är att nå ut till allmänheten för att ge dem en bättre förståelse om vad syntetisk biologi faktiskt handlar om och vad man kan göra, säger Patricia Gullberg som deltog i LiU iGEM-laget 2014 och är dessutom handledare för årets lag.
- Det handlar även om att dela kunskap mellan varandra inom forskningsvärlden, berättar Patricia. Just nu läser hon masterprogrammet Protein Science vid LiU efter att hon examinerades från KBnv. Vi fick tag på henne för en intervju där hon berättar om sina erfarenheter i iGEM. 


Vad är läget för årets lag?
- Laget har precis blivit sammansatt och just nu hålls många möten för att få in de nya deltagarna i projektet då vi har valt att fortsätta på samma spår som förra året, men förmodligen med en del förändringar. Vi kommer även jobba mycket med att brainstorma, planera och förbereda mycket inför sommarens arbete den närmaste tiden.

Vad fick dig att vilja vara med? Vad fick dig att även vara med ett andra år?
- Jag tyckte att tävlingen lät väldigt rolig och givande och jag såg en chans att utveckla mina kunskaper ännu mer. Jag tyckte också det verkade väldigt häftigt med att åka till Boston för att tävla och träffa massa andra människor som brinner för samma sak. Att jag valde att vara med ytterligare ett år till beror på att det var så sjukt kul och givande och jag brinner för projektet så jag kunde inte riktigt släppa det. När jag blev erbjuden en anställning som handledare kunde jag inte tacka nej.

Härligt! Vad har du fått ut av tävlingen?
- Jag har fått många nya goda vänner, jag har utvecklats som person och känner mig självsäker på lab på ett helt annat sätt nu. Då detta är ett studentdrivet projekt är det vi studenter som gör allting från grunden. Då man tvingas utmana sig själv och läsa på för att förstå hur man ska lösa ett problem så utvecklar man verkligen sitt problemtänkande och man vågar ta för sig mer.

Sammanfattningsvis, vad är det bästa med iGEM?
- Att man får lära känna nya människor och utnyttja sina kunskaper på ett väldigt stimulerande sätt!



För mer information:

iGEMs oficiella hemsida
https://www.igem.org/Main_Page

Förra årets bidrag LiUiGEM 2014
http://2014.igem.org/Team:Linkoping_Sweden

Artikel i lokaltidningen Corren
http://www.corren.se/nyheter/linkoping/biologistudenter-vill-hjalpa-jordnotsallergiker-7151455.aspx


\newssep

Nytt studenthus 2018

Drömmen om ett nytt studenthus, med ännu fler studieplatser och där all service för studenterna är samlad, kommer nu att gå i uppfyllelse. På platsen där nuvarande Origo står planeras ett sexvåningshus stå klart för inflyttning september 2018.

Målet är att studenthuset ska bidra till ett större samarbete mellan universitetet och samhället men även samla servicefunktioner för studenterna. Istället för att servicefunktionerna är utspridda på sju olika hus, som det är i nuläget, kommer exempelvis reception, studenthälsan och antagnings- och examensservice samlas på en och samma plats. Det kommer även finnas fler studieplatser, ett studentfik, salar till konferenser eller föredrag och ett sprillans nytt bibliotek. Ett hus för studenterna helt enkelt!
För att öka samarbetet mellan universitetet och samhället kommer en viss verksamhet från Mjärdevi Science Park flytta till Campus Valla. Huset planeras bli byggt med innovativa lösningar för att uppfylla höga krav på energieffektivitet. Det gamla huset Origo kommer att monteras ned på ett hållbart sätt där stora delar av huset kan återbrukas. Satsningen kommer att gå på ca 400 miljoner kronor och kommer verkligen visa att universitetet ligger i framkant och är ett attraktivt val för nya studenter.
 
\image{img/studenthus_inne.jpg}{Inne.}
\image{img/studenthus_ute.jpg}{Ute.}

\newssep

\begin{editorial}{2}{Alumniutskottet}{Alumnitidningen producerades av}{label}
\image{img/alumniutskottet.png}{\\
\noindent Toppen av pyramid: Omid Philip Thern. \\

\noindent Andra våningen från vänster: Elisabet Bengts, \\
Victoria Isabelle Boman. \\

\noindent Nedre våningen från vänster: Julia Bjers, \\
Annie von Scheele, Linda Gidlöf. \\

\noindent I förgrunden från vänster: Elina Jansson, \\
Lovisa Hemmingsson.}
\end{editorial}





\end{document}
